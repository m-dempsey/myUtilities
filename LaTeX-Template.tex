%\documentclass{IEEEtran}
\documentclass[twocolumn]{article}
\usepackage[margin=.75in]{geometry} % % adjust margins and paper size
\usepackage{graphicx} % For including images
\usepackage{setspace} % adjust line spacing
\usepackage{hyperref} % add hyperlinks
\usepackage{authblk} % author block formatiing
\usepackage{amsmath} % math symbols
\usepackage{fancyhdr} % header/footer
\usepackage{lipsum} % dummy text

\pagestyle{fancy} % activate fancy head/foot
\fancyhf{} % clears default header and footer
\fancyhf[C]{UNCLASSIFIED} % add classification head/foot
\fancyhead[L]{My Title}
\fancyhead[R]{DRAFT \today}
\fancyfoot[L]{Distro X: approved for release \\ to Blah Blah Blah} % add distro statement
\fancyfoot[R]{\thepage} % position page number


\begin{document}
\setstretch{1.1}

\title{My Title}
%\author{Mr. Matthew Dempsey, NSWCDD B51\\Mr. Kenton Young, NSWCDD B51}
%\author{Mr. Matthew Dempsey}
%\email{matthew.c.dempsey3.civ@us.navy.mil}
%\author{Mr. Kenton Young}
%\email{kenton.d.young.civ@us.navy.mil}
%\affiliation{NSWCDD, B51}

\author[1]{
    Mr. Matthew Dempsey
    \thanks{matthew.c.dempsey3.civ@us.navy.mil}
    }
\author[1]{
    Mr. John Appleseed
    \thanks{Johnathan.d.Appleseed.civ@us.navy.mil}
    }
\affil[1]{Naval Surface Warfare Center, Dahlgren Division, Code B51}

\begin{titlepage}
    \centering
    \pagenumbering{roman}
    
    \maketitle

    \thispagestyle{fancy}
    
    \begin{abstract}

        \limsup[1]
            
    \end{abstract}

    \paragraph{Keywords:} 
    Here, Are, Some, Keywords

    \paragraph{Distribution Statement:}
    \lipsum[1]

    \newpage

    \tableofcontents
    \listoffigures
    \listoftables

\end{titlepage}

\pagenumbering{arabic}

\section{Introduction}

Establishes the context, introduces the problem, and explains why the reader should care.

\lipsum[1]

\section{Background}

In-depth information about the problem, often including data, research.

\lipsum[1-2]

\section{Methodology}

Here is what we did. Describe general process?
Logical order of paper:
\begin{enumerate}
    \item thing one
    \item thing two
    \item \vdots
    \item last thing
\end{enumerate}

\lipsum[1]

\subsection{First Meaty Section}
\label{sec:OpeningSection}

\limsup[1-2]

\subsection{Another Section}
\label{sec:AnotherSection}

\limsup[1-2]

\begin{figure}[h]
    \centering
    \includegraphics[width=\linewidth]{Path/To/Image.png}
    \caption{Here is a caption.}
    \label{fig:SampleFigure}
\end{figure}

After observing the phenomenon (as seen in Figure \ref{fig:SampleFigure}), we did some math.
In-line math looks like this $f(t) = \int_{0}^{\infty} \frac{x}{2} dt$.
And equations look like this
\begin{equation}
    \label{eq:sampleEquation}
    p(t) = \frac{f(t)}{t},
\end{equation}
or they can be unnumbered like this
\begin{equation*}
    \label{eq:unnumberedEquation}
    f(t) = \int_{0}^{t} dt O(t),
\end{equation*}
and they can be referenced like equation \ref{eq:sampleEquation} by default or \eqref{eq:sampleEquation} when using \verb|amsmath|.
We can also use the \verb|\begin{align}| like this
\begin{align}
    \label{eq:ExampleAlign}
    2x + 3y &= 7 & 4x - y &= 1
\end{align}
or \verb|\begin{multline}| like this
\begin{multiline}
    \label{eq:long-formula}
    A = (x_1^2 + x_2^2 + \dots + x_n^2) + (y_1^2 + y_2^2 + \dots + y_n^2) \\
        + (z_1^2 + z_2^2 + \dots + z_n^2) - 2(x_1y_1 + x_2y_2 + \dots + x_ny_n) \\
        - 2(x_1z_1 + x_2z_2 + \dots + x_nz_n) - 2(y_1z_1 + y_2z_2 + \dots + y_nz_n) .
\end{multiline}

\section{Conclusions}

\limsup[1-3]
    
\appendix
\section{References}


\end{document}